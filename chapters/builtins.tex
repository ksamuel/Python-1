
\chapter{Builtins}

\Glspl{builtin} are those \glspl{callable}, mostly functions, that are available without the need for an import, such as \lstinline{sorted()} or \lstinline{len()}. They are all in the \lstinline{__builtins__} module (which is always automatically imported and available), and if you call \lstinline{dir()} on it, you'll some have been added and removed Python 3. Others are still here, but their behavior changed.

We are going to cover common and less common traps, assuming you are porting from Python 2.7 as recommanded in the introduction. Again, a lot of this can be helped with tooling, as we'll see in later chapters.

\section{Summary}

\begin{labeling}{summary}
\item [range() and xrange()]: \lstinline{xrange()} removed. Use \lstinline{range()}.
\item [map() and filter()]: changed behavior. Use comprehension lists.
\item [reduce()]: moved. Conditionally import from functools.
\item [input() and raw\_input()]: \lstinline{raw\_input()} removed. Alias it in V2 and use \lstinline{input()} everywhere.
\item [zip() and enumerate()]: now return generators.
\item [long()]: removed. Use \lstinline{int()}.
\item [apply()]: removed. Use unpacking.
\item [execfile()]: removed. Use \lstinline{exec(compile(open("some_module.py").read(), "some_module.py", 'exec'))}
\item [reload()]: moved. Conditionlly import it from \lstinline{importlib} and \lstinline{imp}
\item [I/O, bytes and strings]: big topic. See I/O chapter.
\item [coerce()]: deprecated. Remove.
\item [intern()]: moved. Conditionally import from \lstinline{sys}.
\item [callable()]: removed and added back. Use \lstinline{isinstance(the_thing_you_test, collections.Callable)}.
\end{labeling}

\section{range() and xrange()}

In Python 2, we had two functions, \lstinline{range()} and \lstinline{xrange()}, to generate a sequence of numbers. \lstinline{range()} produced a list, and \lstinline{xrange()} produced some kind of iterable lazy object that does the same, but without pre-computing all the values:

\begin{py2}
>>> range(3)
[0, 1, 2]
>>> for x in range(3):
...     print(x)
...
0
1
2
>>> xrange(3)
xrange(3)
>>> for x in xrange(3):
...     print(x)
...
...
0
1
2
\end{py2}

The difference is that if you do \lstinline{range(10000000000000)} and \lstinline{xrange(10000000000000)}, the first one will probably result in a \lstinline{MemoryError}, trying to fit all those numbers in RAM, while the second one will work like a charm. In any case, \lstinline{xrange()} is usually the most performant choice.

You can use a \lstinline{for} loop on both, so the added value of \lstinline{range()} was limited for the situations where you wanted to do something like slicing or using \lstinline{append()}. For this reason, \lstinline{xrange()} disapeared in Python 3, and \lstinline{range()} was set to behave like \lstinline{xrange()}.

The easy solution is to use the tools we'll talk about later to have the same \lstinline{range()} everywhere. The manual fix, however, is to replace all \lstinline{xrange()} with \lstinline{range()} and pay the performance price in Python 2 if you wish to keep the code base working with both versions. In the rare case you do need a list, convert the returned value:

\begin{py2and3}
>>> list(range(10))
[0, 1, 2, 3, 4, 5, 6, 7, 8, 9]
\end{py2and3}

\section{map() and filter()}

\lstinline{map()} and \lstinline{filter()} are simplified versions of functional programming primitives. \lstinline{map()}, in Python 2, applies a function to all elements of an \gls{iterable}, and returns a new list with the result. \lstinline{filter()} also applies a function to all elements, but returns a new list containing only the elements for which the function returned \lstinline{True}.

E.G, in Python 2:

\begin{py2}
>>> numbers = range(10)
>>> numbers
[0, 1, 2, 3, 4, 5, 6, 7, 8, 9]
>>> def is_even(num):
...     return num % 2 == 0
...
>>> filter(is_even, numbers)
[0, 2, 4, 6, 8]
>>> def power_of_2(num):
...     return num * num
>>> map(power_of_2, numbers)
[0, 1, 4, 9, 16, 25, 36, 49, 64, 81]
\end{py2}

In Python 3, they return an object that looks like a \gls{generator}:

\begin{py3}
>>> filter(is_even, numbers)
<filter object at 0x7f7b39abfa58>
>>> list(filter(is_even, numbers))
[0, 2, 4, 6, 8]
\end{py3}

This means better perfs, but unfortunalty also that you can read them only once, and you can't index them:

\begin{py3}
>>> filter(is_even, numbers)[0]
Traceback (most recent call last):
  File "<stdin>", line 1, in <module>
TypeError: 'filter' object is not subscriptable
>>> even_numbers = filter(is_even, numbers)
>>> list(even_numbers)
[0, 2, 4, 6, 8]
>>> list(even_numbers)
[]
\end{py3}

If you don't use any of the list behavior and just loop, you can let the code as-is. However, if you use any list behavior like slicing, indexing, or calling \lstinline{append()}, the you should convert the result to a list before, and store it in a variable:

\begin{py2and3}
>>> even_numbers = list(filter(is_even, numbers))
\end{py2and3}

In your code, you may also find the user of \lstinline{itertools.imap()} and \lstinline{itertools.ifilter()}. In Python 2 they were used to do the same thing as \lstinline{map()} and \lstinline{filter()}, but with the Python 3 behavior: returning generators. They've been removed, and hence, you should stick to regular \lstinline{map()} and \lstinline{filter()} only. If you are on a code base that runs on Python 2 and 3, Python 2 will have worse performances for that case.

However, idiomatic Python has a tendancy to prefer \glspl{list comprehension}, and since your are converting your code, you could take the opportunity to use them. Filtering and mapping our numbers would go from:

\begin{py2and3}
>>> list(map(power_of_2, filter(is_even, numbers)))
[0, 4, 16, 36, 64]
\end{py2and3}

To:

\begin{py2and3}
>>> [x * x for x in numbers if x % 2 == 0]
[0, 4, 16, 36, 64]
\end{py2and3}

And you can control whever you want this to produce a list or a generator by switching between \lstinline{[]} and \lstinline{()}. It solves all your problems at once: compatibility, readability and performance.

In fact, for this very particular problem, tooling will produce something less efficient than this. So embrace comprehension lists!

\section{reduce()}

\lstinline{reduce()} is another simplified functional programming primitive that helps you apply a function to two first elements of an iterable, get the result, then apply the function to the result and the next element and so on. E.g: if you wish to multiply all numbers from 1 to 10, you could do:

\begin{py2}
>>> def multiply(a, b):
...     return a * b
>>> reduce(multiply, range(1, 11))
3628800
\end{py2}

Guido Van Rossum, the creator of Python, felt like reduce was somewhat complicated to use and deserved an import. Hence, in Python 3 it is only in the \lstinline{functools} module.

There is an easy fix, as starting from Python 2.6 the function is in the builtins AND in the the \lstinline{functools} module. So all you have to do is:

\begin{py2and3}
from functools import reduce
\end{py2and3}

Once again, tooling we'll introduce later will have an automated fix for it.

\section{zip() and enumerate()}

{zip()} and enumerate() are one of those underused handy little tools. Because Python have an automatic \lstinline{for} loop and so no index incrementing, people needed another way to do things like reading two (or more) \glspl{iterable} at the same time (this is what \lstinline{zip()} does) or numbering items (which is \lstinline{enumerate()} specialty):

\begin{py2}
>>> fruits = ["Memberberries", "Gomu Gomu no Mi", "Senzus Beans"]
... is_fruit = [True, True, False]
...
>>> zip(fruits, are_fruits)
[('Memberberries', True), ('Gomu Gomu no Mi', True), ('Senzus Beans', False)]
>>> for f, status in zip(fruits, is_fruit):
...     print(f, status)
...
('Memberberries', True)
('Gomu Gomu no Mi', True)
('Senzus Beans', False)
>>> for num, f in enumerate(fruits):
...     print(num, f)
(0, 'Memberberries')
(1, 'Gomu Gomu no Mi')
(2, 'Senzus Beans')
\end{py2}

In Python 3, they do the same thing, but like \lstinline{map()} and \lstinline{filter()}, return a generator instead of a list. The coping strategy is the same as well: if you just loop once, keep it as-is, if you read it several times, index it, slice it or use list method (\lstinline{append()}, \lstinline{extend()}, etc), convert it to a list before:

\begin{py2and3}
>>> list(zip(fruits, are_fruits)) # same with enumerate()
[('Memberberries', True), ('Gomu Gomu no Mi', True), ('Senzus Beans', False)]
\end{py2and3}

In Python 2, you could find \lstinline{itertools.izip()} as an alternative to \lstinline{zip()} returning a generator. Since it doesn't exist in Python 3, always just use \lstinline{zip()}, and pay the performance penalty on Python 2 in the case you keep a Python 2/3 code base. If it's unacceptable, do a conditional import:

\begin{py2and3}
try:
    from itertools import izip
    zip = izip
except ImportError:
    pass
\end{py2and3}

And use \lstinline{zip()} everywhere as you would do for Python 3.

There is also the matter of \lstinline{itertools.izip_longest()}, a companion function to \lstinline{zip}. In case of looping on two iterables of different length, \lstinline{zip()} stops at the shortest, while \lstinline{itertools.zip_longest()} stops at the longest, filling the missing values with \lstinline{None}.

\lstinline{itertools.izip_longest()} has been renamed \lstinline{itertools.zip_longest()} in Python 3, and so you must conditionally import it:

\begin{py2and3}
try:
    from itertools import izip_longest as zip_longest
except ImportError:
    from itertools import zip_longest
\end{py2and3}

There is no such condiration for \lstinline{enumerate()}. And, of course I'm repeating myself but, we got tools for that.

\section{input() and raw\_input()}

\lstinline{input()} was seriously broken in Python 2. It's supposed to be a way to ask a question to the user and get an answer back, but what it did was to run \lstinline{eval()} on it:

\begin{py2}
>>> type(input(""))
lambda x: x
<type 'function'>
\end{py2}

This is a terrible idea security wise. So \lstinline{raw\_input()} was introduced as a way get user input, but always as a string. However, passing directly a non-ASCII string to \lstinline{raw\_input()} could trigger a \lstinline{UnicodeEncodeError}.

In Python 3, \lstinline{raw\_input()} was removed, and \lstinline{input()} just behaves like \lstinline{raw\_input()}, only it accepts transparently unicode strings.

If you migrate your code to Python 3, just replace all \lstinline{raw\_input()} with \lstinline{input()}. If you used \lstinline{input()} for it's \lstinline{eval()} capabilities, maybe it's time to avoid doing that. But if you must, just call \lstinline{eval()} manually on the result.

However, if you need to support both Python 2 and 3, you need to do something like:

\begin{py2and3}
try:
    raw\_input
except NameError: # python 3
    def eval_input(question):
        """ Use at your own risk ! """
        return eval(input(question))
else: # python 2
    eval_input = input
    input = raw\_input
\end{py2and3}

Then use \lstinline{input()} everywhere, avoid non-ASCII string in the prompt and \lstinline{eval\_input()} when you are really, really sure that's what you want to do. Or not. We will have tools for that, remember?

\section{long()}

Python 2 used to have an additional type for numbers called \lstinline{long}. Any \glspl{literal} have a way to create it with a \lstinline{callable} as well: integers can be created using \lstinline{int()}, and floats using \lstinline{float()}, so the \lstinline{long()} functions existed as well. With the type gone, the function is gone too.

If you used this function, replace it with \lstinline{int()}. Or, you know, let tools doing it for you.

\section{apply()}

\lstinline{apply()}, as the name suggests, applies a function to a list (and optionally a dict) of arguments: it's been deprecated since Python 2.3. If you used it for some reason, replace it with unpacking.

Old way that works in 2.7, but not in 3:

\begin{py2}
>>> def marvelous_function(param1, param2, like_param2_but_better):
...    print(param1)
...    print(param2)
...    print(like_param2_but_better)
...
>>> positional_params = ['First', 'Second']
>>> keyword_argument_params = {"like_param2_but_better": "Best"}
>>> apply(marvelous_function, positional_params, keyword_argument_params)
First
Second
Best
\end{py2}

New way, that works everywhere:

\begin{py2and3}
>>> marvelous_function(*positional_params, **keyword_argument_params)
First
Second
Best
\end{py2and3}

Also, tools and all that. I'm going to stop mentioning it, but you'll keep that in mind, right ?

\section{execfile()}

\lstinline{execfile("some_module.py")} is just a shorter way to do \lstinline{exec(compile(open("some_module.py").read(), "some_module.py", 'exec'))}, so use that. Although if you were doing this instead of just importing the module, there was some black magic going on, and maybe you should investigate.


\section{reload()}

When you import a module twice in Python, the job is done only once and then cached. The second time just fetches the reference from that cache: no file opening, new code parsing, no new execution. It's good for performances, and it means you can be trigger happy with the same import in many files.

But it also means if you are in the process of developping the code that if you change the file on disk and proceed to import it, nothing will happend. You need to restart the Python VM for changes to take effect.

Enter \lstinline{reload()}, a function that will allow you to reload a module without having to stop your Python program or shell. Be careful though, as it doesn't update and references, so all variables pointing to objects from this import before the reload are still holding the hold code: it can be a source of very nasty bugs, and it's why \lstinline{reload()} is not very popular.

It's also why, while it used to be a builtin in Python 2, it now needs an import. Unfortunaly, the import machinery changed between Python 3.3, where you'd have to import \lstinline{imp.reload()} and 3.4 where it's in \lstinline{importlib.reload()}. Of course, if you wish to support Python 2 and 3, you'd do:

\begin{py2and3}
try:
    from importlib import reload # 3.4+
except ImportError:
    from imp import reload # 3.3
\end{py2and3}

I insisted on tooling a few times. Ok, many times. But for this one, be careful. Some tools versions (e.g: 2to3 for Python 2) won't offer automatic transformations for this, other will, but only for \lstinline{imp.reload()} (e.g: 2to3 for Python 3.6 and lower) or \lstinline{importlib.reload()} (e.g: 2to3 for Python 3.7).

Given It's unlikely you call \lstinline{reload()} all over the place, I'd go with manual fixing on this one.

\section{I/O, bytes and strings}

The \lstinline{file()}, \lstinline{buffer()}\lstinline{unicode()} and \lstinline{basestring} \glspl{builtin} were removed. Those ones need to be part of an entire chapter: the topic is notoriously misunderstood and you need to know what you're doing to have a happy migration. See \nameref{chap:text_and_bytes}.

\section{coerce()}

\lstinline{coerce} is a relic of the past, back when Python - version 1.x ! - needed something to facilitate arythmetic operations on different types. It would take two numbers with potentially different types, and convert them to two numbers with similar values, and the same type. E.G:

\begin{py2}
>>> coerce(1, 3.)
(1.0, 3.0)
>>> coerce(1j, 3)
(1j, (3+0j))
\end{py2}

Removed in Python 3, you could emulate it with:

\begin{py2and3}
def coerce(x, y):
    t = type(x + y)
    return (t(x), t(y))
\end{py2and3}

But really, just remove it. I can't think of a single scenario using Python 2 or 3 where it would be useful. No automatic tooling will touch it, and there is no need to really: your \lstinline{DEL} key will do the job fine.


\section{intern()}

\lstinline{intern()} is a rare breed because most people have never heard of it, and never needed it, plus it's pretty low level which is exceptionnal for Python. It's used to tell Python to intern a string, keeping it in a sort of privileged cache allowing for pointer comparison. Practically, if it's used as a key in a dictionary, it will make looking it up the value faster. It the definition of a micro-optimisation, something you don't see often in this language.

In Python 3 it's been moved to \lstinline{sys.intern()}, so you can just import it, or if you need 2/3 compatible code, do a conditional import:

\begin{py2and3}
try:
    from sys import intern
except ImportError:
    pass
\end{py2and3}

But you know what they say: there are lies, damn lies and benchmarks. I've read online people can gain around 25% of perf on dict access, but my tests were closer to 5%. And again, just for one operation, dictionary look up, which is unlikely to be the bottleneck of your application. So make sure you really need it: sometimes the easiest fix is just to remove the code.

\section{callable()}

\lstinline{callable()} is a simple function returning \lstinline{True} if something is a \gls{callable}, like a function or a class. Being rarely used, it's been removed in Python 3, before being introduced again in 3.2.

If you followed the advice of not ever using Python 3.0 or 3.1, you can safely ignore this one. If you did not, you can replace it with \lstinline{isinstance(the_thing_you_test, collections.Callable)}.
