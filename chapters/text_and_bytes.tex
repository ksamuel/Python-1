\chapter{Bytes and text}

% http://www.dabeaz.com/python3io_2010/MasteringIO.pdf

Text strings in Python 3 require either 2x as much memory to store as Python 2


bad filenames were easy to create in python 2

If you ever see a \udcxx character, it means that a non-decodable byte was passed in from a system interface

s.decode('utf-8','surrogateescape')

TextIOWrapper 10 times faster than codecs.open

% has been removed and reinstroduced

\section{buffer() and memoryview()}

Those two functions respectively create objects of the same name - a \lstinline{buffer} and a \lstinline{memoryview}. Both are a way to get a subset of something without copying it:

\begin{py2}
>>> donkey_lines = "Are we there yet ?\n" * 10000
>>> # this copies the data into a new object:
>>> for x in donkey_lines[:6]:
...    print(x)
A
r
e

w
e
>>> # those don't:
>>> for x in buffer(donkey_lines, 0, 5):
...    print(x)
A
r
e

w
e
>>> for x in memoryview(donkey_lines)[:5]:
...    print(x)
A
r
e

w
e
\end{py2}

It works on \lstinline{bytes}, \lstinline{bytearray}, \lstinline{array.array}...Everything that implements the so-called \textquote{buffer protocol}. This is a very nice optimization that can save quite a lot of memory/CPU if you manipulate huge chunks of bytes and pass around subset of them, e.g: to files or sockets.

With the rise of performant c libs wrapped in Python (numpy, GUI toolkits, database drivers, etc.), the need for more information about the underlying data than what \lstinline{buffer()} was offering became important. \lstinline{memoryview} provides an answer to that, being able to return the shape, dimension or type or the object behind it.

\lstinline{buffer()} is not more in Python 3, just replace it with \lstinline{memoryview()}. The later exists in Python 2.7, plus it does the same thing, just better. The only difficulty will be that \lstinline{buffer()} accepts \lstinline{unicode()} objects but \lstinline{memoryview()} only accept bytes, and so you'll need to encode them.

So:

\begin{py2}
from imaginary_lib import get_tps_reports, send_those_bytes

# Imaginary code returning a lot of bytes
tps_reports = get_tps_reports()

# Imaginary code writting those bytes to a sockets by chunks
# of 1Mio
step = 3
stop = len(tps_reports)
for i in range(0, stop, step):
    send_those_bytes(buffer(tps_reports, i, step))

\end{py2}
