
\chapter{A quick word on how to use this book}

This book assumes you already have an opinion about porting your code to Python 3, and hence, it will not try to convince you to do so.

Instead, the book will be divided in two parts.

The first one lists and explains as much differences as it can between version 2 and 3. This will allow you not only to understand what's going on, but also give you a better overview of the cost a migration might have for your particular case.

The second one describes a step by step strategy, with suggested tooling, to actually performs the migration.

If you read the content from start to finish, it may seem a lot of work. Fortunatly, most projects  don't actually need to apply half of the advices given here. The vast majority of code bases can be just tinkered with, hacking on it until it works, sampling solutions from these pages on the way.

If you are in a rush, a surprisingly high number of projects can get away with just the instructions from the chapter \textquote{Low hanging fruits}.

\begin{warning}
    Nevertheless, you are expected to be confortable with the Python language and its ecosystem. In particular, you should know how to use pip and virtualenv, be familiar with the general syntax of the language, and have no trouble reading simple scripts containing a few custom functions and classes. If it's not the case, take some time to reach this stage, as porting Python 2 to Python 3 code is not that complicated, but does get way harder if you don't have a  foundation to support you.
\end{warning}
