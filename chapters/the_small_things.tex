
\chapter{Language differences}

80\% of porting your Python 2 code to Python 3 will be about little details. Thankfully, we'll see later than some of it can be automated or at least made easier by using tools provided by the community, so don't rush trying to change all your code base yet.

\section{To print(), and not to print}

You probably already know this, \lstinline{print} used to be a \gls{keyword}, and is now a builtin function. This is the \textquote{Hello, world} of all Python 2/3 migration guide, they all start by this, and we'll honor the tradition.

For simple display, this mean we go from this:

\begin{py2}
print "Howdy, Earth!"
\end{py2}

To:

\begin{py3}
print("Howdy, Earth!")
\end{py3}

And since \lstinline{print} \href{https://www.python.org/dev/peps/pep-0214/}{does fancy things}:

\begin{py2}
with open('out.log', 'w') as f:
    print >>f, "This redirects to a log file"
print "Ending with a coma skips the line break",
\end{py2}

You may have to use:

\begin{py3}
with open('out.log', 'w') as f:
    print("This redirects to a log file", file=f)
print("Ending with a coma skips the line break", end="")
\end{py3}

Starting from Python 2.7, you can turn on the Python 3 behavior using:

\begin{py2and3}
from __future__ import print_function
\end{py2and3}

At the begining of a file. It will be limited to that file and needs to be repeated.

I advice you to introduce it in your project as early as possible, espacially since the muscle memory from typing the old syntax of \lstinline{print} is probably what will take you the most time to port! Besides, the function version has some additional nice features (see the chapter \textquote{New features to take advantage of}).

If you really don't want to do change all your prints yet, and don't want to use \lstinline{__future__}, you can define:

\begin{py2and3}
printf = getattr(__builtins__, 'print')
print("Look Ma, print() function without __future__!")
\end{py2and3}

But really, \lstinline{__future__} is your friend.

\section{Comparing things}

\lstinline{cmp()} is a bit complicated and needs some explanation.

It's common for a programming language to request a callback that will tell a sorting function if something is greater, smaller or equal. One way of doing it is to expect the function to return 1, -1 or 0, and this is the old way of doing it in Python.

E.G, sorting string by length using the deprecated method:

\begin{py2}
>>> famous_last_words = [
...     "Yippee ki-yay",
...     "Kawabunga",
...     "Kamehameha",
... ]
>>> def compare(first, second):
...     if len(first) > len(second):
...         return 1
...     elif len(first) < len(second):
...         return -1
...     else:
...         return 0
...
>>> sorted(famous_last_words, cmp=compare)
['Kawabunga', 'Kamehameha', 'Yippee ki-yay']
\end{py2}

To make this easier, Python 2 came with the \lstinline{cmp()} builtin, that took 2 elements, compared them, and returned 1, -1 or 0 depending of the result, letting you write:

\begin{py2}
>>> def compare(first, second):
...     return cmp(len(first), len(second))
>>> sorted(famous_last_words, cmp=compare)
['Kawabunga', 'Kamehameha', 'Yippee ki-yay']
\end{py2}

However, a more modern and easy method has been introduced in Python 2.4: the callback now just returns the element on which to sort, and Python will use the natural ordering (numerical, alphabetical, etc) of the object. The new callback must be passed through the \lstinline{key} parameter:

\begin{py2and3}
>>> sorted(famous_last_words, key=len)
['Kawabunga', 'Kamehameha', 'Yippee ki-yay']
\end{py2and3}

Hence, \lstinline{cmp()} has been removed from Python 3, and the \lstinline{cmp} parameter has also been scrapped from the \lstinline{sorted()}, \lstinline{max()} and \lstinline{min()} functions.

You should convert your code to using this easier sorting process, espacially since it works in Python 2.7.

If you can't, there are two things you can do.

First, to workaround the fact \lstinline{cmp()} disapeared, you can write your own \lstinline{cmp()}:

\begin{py2and3}
def cmp(a, b):
    return (a>b)-(a<b)
\end{py2and3}

But remember: in Python 3, \lstinline{a} and \lstinline{b} MUST be of compatible types. E.g: two ints, or a int and a float, or two strings. In Python 2 you would use two incompatible types and it would return something, which most of the time was a mistake. In Python 3 it would raise an error, a saner behavior.

To compensage the removal of the \lstinline{cmp} parameter, you can use \lstinline{functools.cmp_to_key} to pass old style sorting callback as you would do a new style sorting callback:

\begin{py2and3}
>>> from functools import cmp_to_key
>>> sorted(famous_last_words, key=cmp_to_key(compare))
['Kawabunga', 'Kamehameha', 'Yippee ki-yay']
\end{py2and3}

Now the bad news is that there is no way to fix that automatically. You'll need to get your hands dirty. The very bad news is that there is even more to fix, as the related \gls{dunder} method \lstinline{__cmp__} doesn't exist in Python 3. We'll cover this in the chapter on object oriented programming.


\section{Dictionaries}

To iterate on keys, values or key/value pairs, you need to use special methods in Python:

\begin{py2}
>>> population = {
...     "Octopod": 2 - 1,
...     "Xenomorph": "too many",
...     "Goa'uld": 0,
...     "Wookie": "no enough" ,
...     "Andalite": 2,
...     "Mondoshawan": "?",
... }
>>>
>>> print(population.keys())
... print(population.values())
... print(population.keys())
...
['Wookie', "Goa'uld", 'Octopod', 'Xenomorph', 'Andalite', 'Mondoshawan']
['no enough', 0, 1, 'too many', 2, '?']
['Wookie', "Goa'uld", 'Octopod', 'Xenomorph', 'Andalite', 'Mondoshawan']
\end{py2}

Because those methods produce lists, an operation that consume a lot of memory and CPU for a simple iteration, alternatives were provided.

The \lstinline{iter*} methods return a generator, sparing the need to allocate arrays:

\begin{py2}
>>> print(population.iterkeys())
... print(population.itervalues())
... print(population.iterkeys())
<dictionary-keyiterator object at 0x7f56340a0628>
<dictionary-valueiterator object at 0x7f56340a0628>
<dictionary-keyiterator object at 0x7f56340a0628>
\end{py2}

They are \gls{iterable} as well, so you can use a \lstinline{for} loop on them just like with lists, but but they can only be read once:

\begin{py2}
>>> names = population.iterkeys()
>>> list(names)
['Wookie', "Goa'uld", 'Octopod', 'Xenomorph', 'Andalite', 'Mondoshawan']
>>> list(names)
[]
\end{py2}

A better solution was found using memory views:

\begin{py2}
>>> print(population.viewitems())
... print(population.viewvalues())
... print(population.viewkeys())
dict_items([('Wookie', 'no enough'), ("Goa'uld", 0), ('Octopod', 1), ('Xenomorph', 'too many'), ('Andalite', 2), ('Mondoshawan', '?')])
dict_values(['no enough', 0, 1, 'too many', 2, '?'])
dict_keys(['Wookie', "Goa'uld", 'Octopod', 'Xenomorph', 'Andalite', 'Mondoshawan'])

\end{py2}

They save the same amount of resources, but can be read multiple times.

However, in Python 3:

\begin{itemize}
    \item \lstinline{iter*} and \lstinline{view*} methods have been removed;
    \item \lstinline{keys()}, \lstinline{values()} and \lstinline{items()} now return memory views.
\end{itemize}

So when you migrate, you want to replace all \lstinline{iter*} and \lstinline{view*} methods by regular \lstinline{keys()}, \lstinline{values()} and \lstinline{items()}. Obviously, if your code will run both in Python 2 and 3 for a while, the Python 2 code will become less performant. You can create a function to avoid this:

\begin{py2and3}
import sys

if sys.version_info.major < 3:
    def dict_items(d):
        return d.viewitems()
else:
    def dict_items(d):
        return d.items()
\end{py2and3}

And use that instead of calling the methods directly.

But don't be too hasty to do this, as in the second part of the book we will see some lirbaries that already offer those kind of tools.

One other gotcha : memory views cannot be used exactly like lists. E.G: you cannot index them (\lstinline{population.viewitems()[0]} won't work), use \lstinline{append()} on them or add them to another list.

If your code need to do that, convert them to lists before:

\begin{py3}
stats = list(population.viewitems())
\end{py3}

\section{Operations}

Coders doing a lot of maths will quickly notice de division operator has changed.

Where in Python 2 you used to do:

\begin{py2}
>>> 4 / 3
1
>>> float(4) / 3  # or you may see operator.truediv(4, 3)
1.3333333333333333
\end{py2}

In Python 3, the default division results in a float:

\begin{py3}
>>> 4 / 3
1.3333333333333333
>>> 4 // 3  # if you want the old behavior, use this new operator
\end{py3}

Now it's a trap because \lstinline{4 // 3} is valid Python 2, but it's doesn't do the same thing:

\begin{py2}
>>> 4 // 3
1
\end{py2}

Fortunatly, you can activate the behavior of Python 3 in Python 2 by adding at the top of each file:

\begin{py2and3}
from __future__ import division
\end{py2and3}

Like with \lstinline{print()}, I advise you to do this as early as possible.

Along with this change, Python 2 removed two things that I doubt you will miss but it's always good to mention.

\begin{itemize}
    \item The \lstinline{<>} operator. It was exactly like \lstinline{!=}. Just do a search and replace for this, really.
    \item The \lstinline{long} number type. When you created a big integer, Python turned it into a new type which was neither \lstinline{int} nor \lstinline{float}, and appeared suffixed with an L in the terminal (e.g: 9999999999999999999L). This is an implementation detail and you should not worry about it unless you used to parse it manually, in which case, do it in a condition.
\end{itemize}


\section{Syntax}

True = False
% Exception: http://python3porting.com/differences.html#exception-objects

backtick

exec is a statement vs function in python 3

%https://python-future.org/compatible_idioms.html#standard-library

\section{Object oriented programming}

