
\chapter{Language differences}

80\% of porting your Python 2 code to Python 3 will be about little details. Thankfully, we'll see later than some of it can be automated or at least made easier by using tools provided by the community, so don't rush trying to change all your code base just yet.

\section{To print(), and not to print}

You probably already know this, \lstinline{print} used to be a \gls{keyword}, and is now a \gls{builtin} function. It's the \textquote{Hello, world} of all Python 2/3 migration guides, and we'll honor the tradition.

For simple display, this mean we go from this:

\begin{py2}
print "Howdy, Earth!"
\end{py2}

To:

\begin{py3}
print("Howdy, Earth!")
\end{py3}

And since \lstinline{print} \href{https://www.python.org/dev/peps/pep-0214/}{does fancy things}:

\begin{py2}
with open('out.log', 'w') as f:
    print >>f, "This redirects to a log file"
print "Ending with a coma skips the line break",
\end{py2}

You may have to use:

\begin{py3}
with open('out.log', 'w') as f:
    print("This redirects to a log file", file=f)
print("Ending with a coma skips the line break", end="")
\end{py3}

Starting from Python 2.7, you can turn on the Python 3 behavior using:

\begin{py2and3}
from __future__ import print_function
\end{py2and3}

At the begining of a file. It will be limited to that file and needs to be repeated.

I advice you to introduce it in your project as early as possible, espacially since the muscle memory from typing the old syntax of \lstinline{print} is probably what will take you the most time to port! Besides, the function version has some additional nice features (see the chapter \nameref{chap:new_features}.).

If you really don't want to do change all your prints yet, and don't want to use \lstinline{__future__}, you can define:

\begin{py2and3}
printf = getattr(__builtins__, 'print')
print("Look Ma, print() function without __future__!")
\end{py2and3}

But really, \lstinline{__future__} is your friend. This is an easy fix, and tooling catches it very well, the hard part is to get used to it.


\section{Dictionaries}

To iterate on keys, values or key/value pairs, you need to use special methods in Python:

\begin{py2}
>>> population = {
...     "Octopod": 2 - 1,
...     "Xenomorph": "too many",
...     "Goa'uld": 0,
...     "Wookie": "no enough" ,
...     "Andalite": 2,
...     "Mondoshawan": "?",
... }
>>>
>>> print(population.keys())
... print(population.values())
... print(population.keys())
...
['Wookie', "Goa'uld", 'Octopod', 'Xenomorph', 'Andalite', 'Mondoshawan']
['no enough', 0, 1, 'too many', 2, '?']
['Wookie', "Goa'uld", 'Octopod', 'Xenomorph', 'Andalite', 'Mondoshawan']
\end{py2}

Because those methods produce lists, an operation that consume a lot of memory and CPU for a simple iteration, alternatives were provided.

The \lstinline{iter*} methods return a generator, sparing the need to allocate arrays:

\begin{py2}
>>> print(population.iterkeys())
... print(population.itervalues())
... print(population.iterkeys())
<dictionary-keyiterator object at 0x7f56340a0628>
<dictionary-valueiterator object at 0x7f56340a0628>
<dictionary-keyiterator object at 0x7f56340a0628>
\end{py2}

They are \gls{iterable} as well, so you can use a \lstinline{for} loop on them just like with lists, but but they can only be read once:

\begin{py2}
>>> names = population.iterkeys()
>>> list(names)
['Wookie', "Goa'uld", 'Octopod', 'Xenomorph', 'Andalite', 'Mondoshawan']
>>> list(names)
[]
\end{py2}

A better solution was found using memory views:

\begin{py2}
>>> print(population.viewitems())
... print(population.viewvalues())
... print(population.viewkeys())
dict_items([('Wookie', 'no enough'), ("Goa'uld", 0), ('Octopod', 1), ('Xenomorph', 'too many'), ('Andalite', 2), ('Mondoshawan', '?')])
dict_values(['no enough', 0, 1, 'too many', 2, '?'])
dict_keys(['Wookie', "Goa'uld", 'Octopod', 'Xenomorph', 'Andalite', 'Mondoshawan'])
\end{py2}

They save the same amount of resources, but can be read multiple times.

However, in Python 3:

\begin{itemize}
    \item \lstinline{iter*} and \lstinline{view*} methods have been removed;
    \item \lstinline{keys()}, \lstinline{values()} and \lstinline{items()} now return memory views.
\end{itemize}

So when you migrate, you want to replace all \lstinline{iter*} and \lstinline{view*} methods by regular \lstinline{keys()}, \lstinline{values()} and \lstinline{items()}. Obviously, if your code will run both in Python 2 and 3 for a while, the Python 2 code will become less performant. You can create a function to avoid this:

\begin{py2and3}
import sys

if sys.version_info.major < 3:
    def dict_items(d):
        return d.viewitems()
else:
    def dict_items(d):
        return d.items()
\end{py2and3}

And use that instead of calling the methods directly.

But don't be too hasty to do this, as in the second part of the book we will see some libraries that already offer those kind of tools.

One other gotcha: memory views cannot be used exactly like lists. E.G: you cannot index them (\lstinline{population.viewitems()[0]} won't work), use \lstinline{append()} on them or add them to another list.

If your code need to do that, convert them to lists before:

\begin{py3}
stats = list(population.viewitems())
\end{py3}

And one last thing to change for the road: the \lstinline{dict.has_key()} method doesn't exist anymore. You should use the \lstinline{in} \gls{keyword}, which works in Python 2 and 3.

\section{Comparing things}

\subsection{Comparison operators}

In Python 2.7, it was allowed to compared any object, even when it didn't make sense, and it was the gateway for bugs. Consider the following snippet:

\begin{py2}
>>> max_weight = 800
>>> weight = 790
>>> weight < max_weight
True
>>> weight = raw_input("How much ?")
How much ?790
>>> weight < max_weight
False
\end{py2}

The problem here is that \lstinline{raw_input()} always returns a string. So we are not comparing \lstinline{800} to \lstinline{790}, we are comparing \lstinline{800} to the string \lstinline{"790"}, and comparing numbers and text makes no sense. But Python tries to do it, and get it wrong. The proper code should be:

\begin{py2}
>>> int(weight) < max_weight
True
\end{py2}

This is one of the worst possible bug: a silent one, and that seems to work most of the time. Very hard to debug. So the behavior in Python 3 when comparing incompatible types is now to raise an error:

\begin{py3}
>>> weight < max_weight
TypeError: '>' not supported between instances of 'str' and 'int'
\end{py3}

This lets you find the bug quickly and fix your logic. That's the nice thing about it: as soon as this code run in Python 3, you will know it has a problem since it will crash, and all you have to do is convert one object to the same type as the other. Bottom line, if you see a \lstinline{TypeError} near a \lstinline{<}, \lstinline{<=}, \lstinline{>=}or \lstinline{>}, make sure that you are comparing compatible types. Tooling won't help much with that.

Alternatively, you can define (or update if you were already doing so) special methods on one of the objects so that they can now be compared, as seen in the chapter \nameref{chap:oop}.

One last thing: there used to be a \lstinline{<>} operator in Python 2.7. It was doing the same job as \lstinline{!=}, and was redundant. It doesn't exist in Python 3, you can replace it with \lstinline{!=} everywhere and tools will do that for you.

\subsection{Comparison functions}

The story is a bit more complicated and needs a detailed explanation.

It's common for a programming language to request a callback that will tell a sorting function if something is greater, smaller or equal. One way of doing it is to expect the function to return 1, -1 or 0, and this is the old way of doing it in Python.

E.G, sorting string by length using the deprecated method:

\begin{py2}
>>> wise_latin_quotes = [
...     "Yippee ki-yay",
...     "Kawabunga",
...     "Kamehameha",
... ]
>>> def compare(first, second):
...     if len(first) > len(second):
...         return 1
...     elif len(first) < len(second):
...         return -1
...     else:
...         return 0
...
>>> sorted(wise_latin_quotes, cmp=compare)
['Kawabunga', 'Kamehameha', 'Yippee ki-yay']
\end{py2}

This mechanism exists for \lstinline{sorted()}, but also on the \lstinline{list.sort()} method, and on \lstinline{min()} and \lstinline{max()} functions.

To easy writting this kind of code, Python 2 came with the \lstinline{cmp()} \gls{builtin}, that took two elements, compared them, and returned 1, -1 or 0 depending of the result, letting you write:

\begin{py2}
>>> def compare(first, second):
...     return cmp(len(first), len(second))
>>> sorted(wise_latin_quotes, cmp=compare)
['Kawabunga', 'Kamehameha', 'Yippee ki-yay']
\end{py2}

Feel like you are seeing double ? Don't confuse \lstinline{cmp()}, the function that returns 1, -1 or 0, and \lstinline{cmp}, the parameter you need to pass the callback (here \lstinline{compare()}) to. They have the same name, but they do completly different things!

In fact, this \lstinline{cmp} parameter has been subject to change too, albeit a different change.

Indeed, a more modern and easy method for providing a callback for comparing object has been introduced in Python 2.4: the function should now just return the element on which to sort, and Python will use the natural ordering (numerical, alphabetical, etc) of the returned object. This new callback must be passed through the \lstinline{key} parameter instead of \lstinline{cmp}. Here is the same code as before, but with this new, easier method:

\begin{py2and3}
>>> sorted(wise_latin_quotes, key=len)
['Kawabunga', 'Kamehameha', 'Yippee ki-yay']
\end{py2and3}

This new way of comparing became the standard and recommanded one, so the \lstinline{cmp()} function has been removed from Python 3, and the \lstinline{cmp} parameter has also been scrapped.

You should convert your code to using this easier sorting process, espacially since it works in Python 2.7.

If you can't, there are two things you can do.

First, to workaround the fact \lstinline{cmp()} disapeared, you can write your own \lstinline{cmp()}:

\begin{py2and3}
def cmp(a, b):
    return (a>b)-(a<b)
\end{py2and3}

But remember: in Python 3, \lstinline{a} and \lstinline{b} MUST be of compatible types. E.g: two ints, or a int and a float, or two strings. In Python 2 you would use two incompatible types and it would return something, which most of the time was a mistake. In Python 3 it would raise an error.

To compensate the removal of the \lstinline{cmp} parameter, you can use \lstinline{functools.cmp_to_key} to pass any old style sorting callback you still have directly to \lstinline{key}:

\begin{py2and3}
>>> from functools import cmp_to_key
>>> sorted(wise_latin_quotes, key=cmp_to_key(compare))
['Kawabunga', 'Kamehameha', 'Yippee ki-yay']
\end{py2and3}

Now the bad news is that there is no way to fix that automatically. You'll need to get your hands dirty. The very bad news is that there is even more to fix, as the related \gls{dunder} method \lstinline{__cmp__} doesn't exist in Python 3. We'll cover this in the chapter \nameref{chap:oop}.

\section{Exceptions}

All those are pretty old, and well supported by automated tooling. Still, I encoutered a few of them in the wild, so it's good to know.

\subsection{Raising}

Old Python versions allowed to raise an error from any object, not just exceptions. As a result, it was possible to do something like:

\begin{py2}
raise "Woops!"
\end{py2}

Another alternative syntax was:

\begin{py2}
raise TypeError, "Woops!"
\end{py2}

Both of those are forbidden now, and you should convert them to the standard syntax were you instanciate an exception class (any class that inherit from \lstinline{BaseException}) and pass it the arguments:

\begin{py2and3}
raise TypeError("Woops!")
\end{py2and3}

If you have custom exception classes, make sure they inherit from \lstinline{Exception} or one of its children. \lstinline{Exception} also replaces \lstinline{StandardError} that has been removed.

\subsection{Catching}

This one is sneaky. Look at this code:

\begin{py2}
try:
    1 / 0
except ImportError, ZeroDivisionError:
    return 42
\end{py2}

It looks like it should return 42 happily, but it crashes with \lstinline{ZeroDivisionError: integer division or modulo by zero}.

In Python 2, what this code does is catching \lstinline{ImportError}, and puting the result in a variable named \lstinline{ZeroDivisionError}. This code is actually doing:

\begin{py2}
try:
    1 / 0
except ImportError as ZeroDivisionError:
    return 42
\end{py2}

While what we meant is:

\begin{py2}
try:
    1 / 0
except (ImportError, ZeroDivisionError):
    return 42
\end{py2}

So, you should convert any of those occurences to the more modern idioms. Besides, the old way is a syntax error in Python 3.

Also, a subtil detail that caught me by surprise once was that exceptions used to be \gls{iterable}. In Python 2, you could do:

\begin{py2}
>>> error = AssertionError("You do!", "I don't!", "You do!", "I don't!", "You don't.", "I do!",  "You don't!", "I do!")
>>> for line in error:
...     print(line)
You do!
I don't!
You do!
I don't!
You don't.
I do!
You don't!
I do!
\begin{py2}

In Python 3 they don't, you need to iterable on the \lstinline{args} attribute:

\begin{py2and3}
>>> for line in error.args:
...     print(line)
You do!
I don't!
You do!
I don't!
You don't.
I do!
You don't!
I do!
\begin{py2and3}

Granted, it's not something you do everyday.

\subsection{Division}

Coders doing a lot of maths will quickly notice the division operator has changed.

Where in Python 2 you used to do:

\begin{py2}
>>> 4 / 3
1
>>> float(4) / 3  # or you may see operator.truediv(4, 3)
1.3333333333333333
\end{py2}

In Python 3, the default division results in a float:

\begin{py3}
>>> 4 / 3
1.3333333333333333
>>> 4 // 3  # if you want the old behavior, use this new operator
\end{py3}

Now it's a trap because \lstinline{4 // 3} is valid Python 2, but it's doesn't do the same thing:

\begin{py2}
>>> 4 // 3
1
\end{py2}

Fortunatly, you can activate the behavior of Python 3 in Python 2 by adding at the top of each file:

\begin{py2and3}
from __future__ import division
\end{py2and3}

Like with \lstinline{print()}, I advise you to do this as early as possible, espacially since tooling can help little with this.


\section{Reserved keywords and constants}

Adding \glspl{keyword} to a language is a delicate process, since suddendly it becomes a syntax error to use it for anything else. Same with \gls{builtin} constants.

In 2.4 \lstinline{None} became a constant. In 2.6, \lstinline{as} and \lstinline{with} became keywords. In Python 3.0, \lstinline{True} and \lstinline{False} became constants. Yes, in Python 2.7 you could do:

\begin{py2}
True = False
\end{py2}

To mess around with your interns. And vice versa. Fun times.

And finally, in Python 3.6, \lstinline{async} and \lstinline{await} became keywords as well.

So if you have any variable using those names, time to rename them.

\section{Get the parenthesis right}

In Python 2 it was possible to do \lstinline{[daily_state for daily_state in False, False, "FileNotFound"]} in comprehension lists. Notice the implicit tuple at the end. This magic is gone, and you now need to use parenthesis to remove the ambiguity: \lstinline{[daily_state for daily_state in (False, False, "FileNotFound")]}.

Please note that it is just for list comprehensions, and it still possible to create tuples without parenthesis elsewhere:

\begin{py2and3}
>>> daily_state = False, False, "FileNotFound"
>>> daily_state
(False, False, 'FileNotFound')
>>> type(daily_state)
tuple
\end{py2and3}

This works because the operator to create tuples is the coma, not the parenthesis. Parenthesis are only here to remove ambiguity.

Speaking about parenthesis, Python 2 had a nifty feature allowing to unpack variable directly in a function signature:

\begin{py2}
>>> def add_restaurant(name, stars, (lat, long)):
...    print(name)
...    print(stars)
...    print(lat)
...    print(long)
\end{py2}

Which mean you could call the function that way:

\begin{py2}
>>> name = "Les pates vivantes"
>>> stars = 4
>>> coordinates = [48.874732, 2.341594]
>>> add_restaurant(name, stars, coordinates)
Les pates vivantes
4
48.874732
2.341594
\end{py2}

And it's no longer possible in Python 3. You will have to change that manually.

\section{Other syntaxic changes}

\subsection{Octal notation}

Python got a handful of alternative notation to create types. You can create a string doing \lstinline{'tnetennba'}, \lstinline{"tnetennba"}, \lstinline{"""tnetennba"""} or {'''tnetennba'''}. Four different syntaxes, but they result in the same object.

There is something similar for integer. Indeed, you can create the number 175 using \lstinline{175}, but also with \lstinline{0xAF}

Octal literals are no longer of the form 0720; use 0o720 instead. In

\begin{py2}
>>> 0xAF # hexadecimal notation
175
>>> 0b10101111 # binary notation
175
>>> 0257 # octal notation
175
\end{py2}

Again, different notations, but they all create the same number in memory.

It comes handy in specific engeenering fields, where they are the default way of representing some values.

The octal notation changed in Python 3 though, and you must insert a \lstinline{o} as the second charater to avoid the syntax error:

\begin{py2and3}
>>> 0o257
\begin{py2and3}

It works in Python 2.7 as well. Just update to the most recent syntax.

\subsection{repr()}

\lstinline{repr()} is a what the console use when you type in an object without printing it:

\begin{py2and3}
>>> "Wololo"
'Wololo'
>>> print("Wololo")
Wololo
>>> print(repr("Wololo"))
'Wololo'
\end{py2and3}

It tries to return something you can pass to \lstinline{exec()}, and if it can't, it will return something like \lstinline{<ClassName>}.

We used to be able to use backsticks (\lstinline{`}) as a shorthand to get an object \lstinline{repr()}:

\begin{py2}
>>> print(`"Wololo"`)
'Wololo'
\end{py2}

This is not possible anymore. Replace it.



exec is a statement vs function in python 3

%https://python-future.org/compatible_idioms.html#standard-library


% u"test" removed and added back
% L remvoed

% The from module import * syntax is only allowed at the module level, no longer inside functions.

% Removed keyword: exec() is no longer a keyword; it remains as a function. (Fortunately the function syntax was also accepted in 2.x.) Also note that exec() no longer takes a stream argument; instead of exec(f) you can use exec(f.read())

